% Options for packages loaded elsewhere
\PassOptionsToPackage{unicode}{hyperref}
\PassOptionsToPackage{hyphens}{url}
%
\documentclass[
]{book}
\usepackage{lmodern}
\usepackage{amssymb,amsmath}
\usepackage{ifxetex,ifluatex}
\ifnum 0\ifxetex 1\fi\ifluatex 1\fi=0 % if pdftex
  \usepackage[T1]{fontenc}
  \usepackage[utf8]{inputenc}
  \usepackage{textcomp} % provide euro and other symbols
\else % if luatex or xetex
  \usepackage{unicode-math}
  \defaultfontfeatures{Scale=MatchLowercase}
  \defaultfontfeatures[\rmfamily]{Ligatures=TeX,Scale=1}
\fi
% Use upquote if available, for straight quotes in verbatim environments
\IfFileExists{upquote.sty}{\usepackage{upquote}}{}
\IfFileExists{microtype.sty}{% use microtype if available
  \usepackage[]{microtype}
  \UseMicrotypeSet[protrusion]{basicmath} % disable protrusion for tt fonts
}{}
\makeatletter
\@ifundefined{KOMAClassName}{% if non-KOMA class
  \IfFileExists{parskip.sty}{%
    \usepackage{parskip}
  }{% else
    \setlength{\parindent}{0pt}
    \setlength{\parskip}{6pt plus 2pt minus 1pt}}
}{% if KOMA class
  \KOMAoptions{parskip=half}}
\makeatother
\usepackage{xcolor}
\IfFileExists{xurl.sty}{\usepackage{xurl}}{} % add URL line breaks if available
\IfFileExists{bookmark.sty}{\usepackage{bookmark}}{\usepackage{hyperref}}
\hypersetup{
  pdftitle={CH10009 Workshop Questions},
  pdfauthor={Fiona Dickinson},
  hidelinks,
  pdfcreator={LaTeX via pandoc}}
\urlstyle{same} % disable monospaced font for URLs
\usepackage{longtable,booktabs}
% Correct order of tables after \paragraph or \subparagraph
\usepackage{etoolbox}
\makeatletter
\patchcmd\longtable{\par}{\if@noskipsec\mbox{}\fi\par}{}{}
\makeatother
% Allow footnotes in longtable head/foot
\IfFileExists{footnotehyper.sty}{\usepackage{footnotehyper}}{\usepackage{footnote}}
\makesavenoteenv{longtable}
\usepackage{graphicx,grffile}
\makeatletter
\def\maxwidth{\ifdim\Gin@nat@width>\linewidth\linewidth\else\Gin@nat@width\fi}
\def\maxheight{\ifdim\Gin@nat@height>\textheight\textheight\else\Gin@nat@height\fi}
\makeatother
% Scale images if necessary, so that they will not overflow the page
% margins by default, and it is still possible to overwrite the defaults
% using explicit options in \includegraphics[width, height, ...]{}
\setkeys{Gin}{width=\maxwidth,height=\maxheight,keepaspectratio}
% Set default figure placement to htbp
\makeatletter
\def\fps@figure{htbp}
\makeatother
\setlength{\emergencystretch}{3em} % prevent overfull lines
\providecommand{\tightlist}{%
  \setlength{\itemsep}{0pt}\setlength{\parskip}{0pt}}
\setcounter{secnumdepth}{5}
\usepackage{booktabs}
\usepackage{amsthm}
\makeatletter
\def\thm@space@setup{%
  \thm@preskip=8pt plus 2pt minus 4pt
  \thm@postskip=\thm@preskip
}
\makeatother
\usepackage[]{natbib}
\bibliographystyle{apalike}

\title{CH10009 Workshop Questions}
\author{Fiona Dickinson}
\date{2020-10-12}

\begin{document}
\maketitle

{
\setcounter{tocdepth}{1}
\tableofcontents
}
\hypertarget{welcome}{%
\chapter*{Welcome}\label{welcome}}
\addcontentsline{toc}{chapter}{Welcome}

The notes have been prepared in a package called BookDown for RStudio so that the equations are accessible to screen readers. However, by providing the notes as a .html webpage I can also embed short videos to further describe some of the topics. Further you can download the questions (and later the answers, top left of the screen) in a format that suits you (either pdf or epub) to view offline, or change the way this document appears for ease of reading.

If you spot any typos or think there are any errors please let me know and I will do my best to fix them.

\hypertarget{workshops-for-ch10009}{%
\section*{Workshops for CH10009}\label{workshops-for-ch10009}}
\addcontentsline{toc}{section}{Workshops for CH10009}

The topics for LOILS each week are as follows:

\begin{itemize}
\tightlist
\item
  Week 1: General Q\&A
\item
  Week 2: Rearranging equations, units and standard form
\item
  Week 3: Logarithms and exponentials
\item
  Week 4: Tables and graphs
\item
  Week 5: Calculus - differentiation - the basics and the chain rule
\item
  Week 6: Calculus - differentiation - the product rule and partial differentiation
\item
  Week 7: Calculus - integration - the basics and definite integrals
\item
  Week 8: Some more examples of integration and revision
\end{itemize}

This `book' will be updated weekly with workshop questions, answers will be provided and some answers will include `process' as well as answer. Please contact me if you need help.

I am using this format as it is an accessible format.

\hypertarget{version-history}{%
\section*{Version history}\label{version-history}}
\addcontentsline{toc}{section}{Version history}

Week 2 workshop released 12th October 2020.

Some more video answers for workshop 1 embedded 11th October 2020.

Video answers for workshop 1 embedded 09th October 2020.

The initial commit of this book is dated 2nd October 2020.

\hypertarget{ch:Workshop1}{%
\chapter{Week 1}\label{ch:Workshop1}}

\hypertarget{sec:Prelim}{%
\section{Preliminary infomation}\label{sec:Prelim}}

\hypertarget{si-base-units}{%
\subsection{SI base units}\label{si-base-units}}

The SI system of base units has seven fundamental units from which the others are derived.

\begin{longtable}[]{@{}cccc@{}}
\caption{\label{tab:SIbase} The seven base units from which all others are dervied.}\tabularnewline
\toprule
SI base unit & symbol & quantity symbol (dimension) & quantity\tabularnewline
\midrule
\endfirsthead
\toprule
SI base unit & symbol & quantity symbol (dimension) & quantity\tabularnewline
\midrule
\endhead
kilogram & kg & M & mass\tabularnewline
metre & m & L & length\tabularnewline
second & s & T & time\tabularnewline
ampere & A & I & electric current\tabularnewline
kelvin & K & Θ & temperature\tabularnewline
mole & mol & N & amount of substance\tabularnewline
candela & cd & J & luminous intensity\tabularnewline
\bottomrule
\end{longtable}

\hypertarget{si-derived-units}{%
\subsection{SI Derived units}\label{si-derived-units}}

\begin{longtable}[]{@{}ccccc@{}}
\caption{\label{tab:SIderive} Some common SI derived units used in chemistry.}\tabularnewline
\toprule
symbol & SI derived unit & quantity & SI base units & other SI units\tabularnewline
\midrule
\endfirsthead
\toprule
symbol & SI derived unit & quantity & SI base units & other SI units\tabularnewline
\midrule
\endhead
Hz & hertz & frequency & s\textsuperscript{-1} &\tabularnewline
N & newton & force & kg m s\textsuperscript{-2} &\tabularnewline
Pa & pascal & pressure & kg m\textsuperscript{-1} s\textsuperscript{-2} & N m\textsuperscript{-2}\tabularnewline
J & joule & energy & kg m\textsuperscript{2} s\textsuperscript{-2} & N m\tabularnewline
W & watt & power & kg m\textsuperscript{2} s\textsuperscript{-3} & J s\textsuperscript{-1}\tabularnewline
C & coulomb & electrical charge & A s &\tabularnewline
V & volt & electrical potential & kg m\textsuperscript{2} s\textsuperscript{-3} A\textsuperscript{-1} & J C\textsuperscript{-1}\tabularnewline
F & farad & electrical capacitance & kg\textsuperscript{-1} m\textsuperscript{-2} s\textsuperscript{4} A\textsuperscript{2} & C V\textsuperscript{-1}\tabularnewline
Ω & ohm & electrical resistance & kg m\textsuperscript{2} s\textsuperscript{-3} A\textsuperscript{-2} & V A\textsuperscript{-1}\tabularnewline
S & siemens & electrical conductance & kg\textsuperscript{-1} m\textsuperscript{-2} s\textsuperscript{3} A\textsuperscript{2} & A V\textsuperscript{-1} or 1/Ω\tabularnewline
\bottomrule
\end{longtable}

\hypertarget{other-units-and-conversion-factors}{%
\subsection{Other units and conversion factors}\label{other-units-and-conversion-factors}}

There are a number of non-SI base or derived units which are in common usage which are useful to know and be able to convert between. Table \ref{tab:nonSI} contains a number of useful unit conversions.

\begin{longtable}[]{@{}ccc@{}}
\caption{\label{tab:nonSI} The relationship between some other common units and the SI system.}\tabularnewline
\toprule
unit & quantity & SI equivalant\tabularnewline
\midrule
\endfirsthead
\toprule
unit & quantity & SI equivalant\tabularnewline
\midrule
\endhead
torr (or mm Hg) & pressure & \(\frac{101325}{760}\) Pa\tabularnewline
atm & pressure & 101325 Pa\tabularnewline
bar & pressure & 100000 Pa\tabularnewline
eV & energy & 1.602176634 × 10\textsuperscript{-19} J\tabularnewline
cal & energy & 4.184 J\tabularnewline
Å & length & 1 × 10\textsuperscript{-10} m\tabularnewline
\bottomrule
\end{longtable}

There are myriad other units in use, either with historical or geographic preference, or just for niche purposes (where would we be without olympic swimming pools or London buses). Examples such as the mile, furlong or beard-second are all units of length.

Further, the unit ºC is formally an SI derived unit. The temperature in Kelvin is:

\begin{equation*}
T (\textrm{K}) = T (\textrm{K}) + 273.15
\end{equation*}

\hypertarget{si-prefixes-and-standard-form}{%
\subsection{SI prefixes and standard form}\label{si-prefixes-and-standard-form}}

In general lower case prefixes are used for negative powers and upper case prefixes are used for positive powers, however k (kilo) is an obvious exception to this rule. (Other exceptions are da (deca, 10\textsuperscript{1}) and h (hecto, 10\textsuperscript{2})).

\begin{longtable}[]{@{}ccc@{}}
\caption{\label{tab:SIprefix} The more common SI prefixes used in chemistry.}\tabularnewline
\toprule
SI prefix & SI prefix `name' & standard form multiplier\tabularnewline
\midrule
\endfirsthead
\toprule
SI prefix & SI prefix `name' & standard form multiplier\tabularnewline
\midrule
\endhead
y & yocto & 10\textsuperscript{-24}\tabularnewline
z & zepto & 10\textsuperscript{-21}\tabularnewline
a & atto & 10\textsuperscript{-18}\tabularnewline
f & femto & 10\textsuperscript{-15}\tabularnewline
p & pico & 10\textsuperscript{-12}\tabularnewline
n & nano & 10\textsuperscript{-9}\tabularnewline
μ & micro & 10\textsuperscript{-6}\tabularnewline
m & milli & 10\textsuperscript{-3}\tabularnewline
c & centi & 10\textsuperscript{-2}\tabularnewline
d & deci & 10\textsuperscript{-1}\tabularnewline
& &\tabularnewline
k & kilo & 10\textsuperscript{3}\tabularnewline
\bottomrule
\end{longtable}

\hypertarget{sec:Questions}{%
\section{Questions}\label{sec:Questions}}

\hypertarget{subsec:rearrange}{%
\subsection{Rearranging equations}\label{subsec:rearrange}}

Answers for these questions are in Section \ref{subsec:rearrangeans}.

For each of the following rearrange to make the specified variable the subject of the equation.

\begin{enumerate}
\def\labelenumi{\arabic{enumi}.}
\tightlist
\item
  \([A]= [A]_0 - kt\), \(t\)
\item
  \(E = \frac{1}{2}mv^2\), \(v\)
\item
  \(F = \frac{q_1 q_2}{4 \pi \varepsilon _0 r^2}\), \(r\)
\item
  \(\frac{1}{[A]}=\frac{1}{[A]_0}+kt\), \([A]_0\)
\item
  \(\ln (x_A)=-\frac{\Delta H}{R}(\frac{1}{T_1}-\frac{1}{T_2})\), \(T_1\)
\item
  \(K_a=\frac{\alpha ^2 c}{1- \alpha}\), \(\alpha\)
\end{enumerate}

\hypertarget{subsec:Unitconv}{%
\subsection{Unit conversion questions}\label{subsec:Unitconv}}

Answers for these questions are in Section \ref{subsec:Unitconvans}.

\begin{enumerate}
\def\labelenumi{\arabic{enumi}.}
\tightlist
\item
  Convert the following:

  \begin{enumerate}
  \def\labelenumii{\alph{enumii}.}
  \tightlist
  \item
    3.4 µm to mm and m
  \item
    270.4 g mol\textsuperscript{-1} to kg mol\textsuperscript{-1} and yg (molecule\textsuperscript{-1})
  \item
    23.4 μg dm\textsuperscript{−3} to mg dm\textsuperscript{−3}, g m\textsuperscript{−3}, and kg m\textsuperscript{−3}
  \item
    17.5 µHz to Hz
  \item
    5796 cm\textsuperscript{-1} to µm\textsuperscript{-1} and m\textsuperscript{-1}
  \end{enumerate}
\end{enumerate}

\begin{center}\rule{0.5\linewidth}{0.5pt}\end{center}

\begin{enumerate}
\def\labelenumi{\arabic{enumi}.}
\setcounter{enumi}{1}
\tightlist
\item
  If a box has dimensions 0.234 m x 34.5 cm x 26.8 mm. What is the volume of the box in:

  \begin{enumerate}
  \def\labelenumii{\alph{enumii}.}
  \tightlist
  \item
    cm\textsuperscript{3}?
  \item
    dm\textsuperscript{3}?
  \item
    m\textsuperscript{3}?
  \item
    Å\textsuperscript{3}?
  \end{enumerate}
\end{enumerate}

\begin{center}\rule{0.5\linewidth}{0.5pt}\end{center}

\begin{enumerate}
\def\labelenumi{\arabic{enumi}.}
\setcounter{enumi}{2}
\tightlist
\item
  The Gibbs free energy of a reaction, \(\Delta G\) is given by equation \eqref{eq:gibbs}.
\end{enumerate}

\begin{equation}
\Delta G = \Delta H - T \Delta S
\label{eq:gibbs}
\end{equation}

Determine the value of \(\Delta G\) at 40 ºC when the enthalpy of reaction, \(\Delta H\) = -10.235 kJ mol\textsuperscript{-1} and the molar entropy, \(\Delta S\) = +34 J K\textsuperscript{-1} mol\textsuperscript{-1}

\hypertarget{subsec:detunits}{%
\subsection{Determining units of variables in equations}\label{subsec:detunits}}

Answers for these questions are in Section \ref{subsec:detunitsans}.

\begin{enumerate}
\def\labelenumi{\arabic{enumi}.}
\tightlist
\item
  The ideal gas equation is given in equation \eqref{eq:idealgas}.
\end{enumerate}

\begin{equation}
pV = n\textrm{R}T
\label{eq:idealgas}
\end{equation}

The units of the variables are:
\(p\) (pressure), Pa (pascals)
\(V\) (volume), m\textsuperscript{3}
\(n\) (number of moles), mol
\(T\) (absolute temperature), K

\begin{enumerate}
\def\labelenumi{\alph{enumi}.}
\item
  Determine the SI base units of the gas constant, R.
\item
  Determine the pressure in mmHg of 1.00 mmol of an ideal gas that occupies 1.65 dm\textsuperscript{3} at 25 ºC.
\end{enumerate}

\begin{center}\rule{0.5\linewidth}{0.5pt}\end{center}

\begin{enumerate}
\def\labelenumi{\arabic{enumi}.}
\setcounter{enumi}{1}
\tightlist
\item
  The famous Einstein equation \(E=mc^2\) is more properly written as:
\end{enumerate}

\begin{equation*}
E^2 = p^2 c^2 + m_0^2 c^4
\end{equation*}

Determine the units of the variable \(p\).

\begin{center}\rule{0.5\linewidth}{0.5pt}\end{center}

\begin{enumerate}
\def\labelenumi{\arabic{enumi}.}
\setcounter{enumi}{2}
\tightlist
\item
  At low temperatures the molar heat capacity of a material \(C_{p, m}\) (J K\textsuperscript{-1} mol\textsuperscript{-1}) is given by equation \eqref{eq:lowtempheat}.
\end{enumerate}

\begin{equation}
C_{p, m}= a T^3
\label{eq:lowtempheat}
\end{equation}

Determine the units of the constant, a.

\begin{center}\rule{0.5\linewidth}{0.5pt}\end{center}

\begin{enumerate}
\def\labelenumi{\arabic{enumi}.}
\setcounter{enumi}{3}
\tightlist
\item
  Determine the units of the coulomb constant, \(k_e\), in equation \eqref{eq:coulomb}, given that r is the separation of two charges, F the force of attraction between the two charges, and \(q_x\) is the charge (in coulombs, C) on each of the particles.
\end{enumerate}

\begin{equation}
F = k_e \frac{q_1 q_2}{r^2}
\label{eq:coulomb}
\end{equation}

\hypertarget{sec:Answers}{%
\section{Answers}\label{sec:Answers}}

\hypertarget{subsec:rearrangeans}{%
\subsection{Rearranging equations answers}\label{subsec:rearrangeans}}

\begin{enumerate}
\def\labelenumi{\arabic{enumi}.}
\tightlist
\item
  \(t =\frac{[A]_0-[A]}{k}\)
\item
  \(v = \sqrt {\frac{2E}{m}}\)
\item
  \(r = \sqrt{\frac{q_1 q_2}{4 \pi \varepsilon_0 F }}\)
\item
  \([A]_0=\frac{[A]}{1-[A]kt}\)
\item
  \(\frac{\Delta H T_2}{\Delta H - R T \ln {x_A}}\)
\end{enumerate}

\begin{enumerate}
\def\labelenumi{\arabic{enumi}.}
\tightlist
\item
  \(\alpha = \frac{-K_a \pm \sqrt{K_a^2 + 4 c K_a}}{2c}\)
\end{enumerate}

\hypertarget{subsec:Unitconvans}{%
\subsection{Unit conversion answers}\label{subsec:Unitconvans}}

\begin{enumerate}
\def\labelenumi{\arabic{enumi}.}
\item
  \begin{enumerate}
  \def\labelenumii{\alph{enumii}.}
  \tightlist
  \item
    3.4 × 10\textsuperscript{-3} mm; 3.4 × 10\textsuperscript{-6} m
  \item
    0.2704 kg mol\textsuperscript{-1}; 449.0 yg
  \end{enumerate}

  \begin{enumerate}
  \def\labelenumii{\alph{enumii}.}
  \setcounter{enumii}{2}
  \tightlist
  \item
    23.4 × 10\textsuperscript{-3} mg dm\textsuperscript{−3}; 23.4 × 10\textsuperscript{-3} g m\textsuperscript{−3}; and 23.4 × 10\textsuperscript{-6} kg m\textsuperscript{−3}
  \item
    17.5 × 10\textsuperscript{-6} Hz
  \item
    0.5796 µm\textsuperscript{-1} and 5.796 × 10\textsuperscript{5} m\textsuperscript{-1}
  \end{enumerate}
\end{enumerate}

\begin{center}\rule{0.5\linewidth}{0.5pt}\end{center}

\begin{enumerate}
\def\labelenumi{\arabic{enumi}.}
\setcounter{enumi}{1}
\item
  \begin{enumerate}
  \def\labelenumii{\alph{enumii}.}
  \tightlist
  \item
    2.16 × 10\textsuperscript{3} cm\textsuperscript{3}
  \item
    2.16 dm\textsuperscript{3}
  \item
    2.16 × 10\textsuperscript{-3} m\textsuperscript{3}
  \item
    2.16 × 10\textsuperscript{27} Å\textsuperscript{3}
  \end{enumerate}
\end{enumerate}

\begin{center}\rule{0.5\linewidth}{0.5pt}\end{center}

\begin{enumerate}
\def\labelenumi{\arabic{enumi}.}
\setcounter{enumi}{2}
\tightlist
\item
  -21 kJ mol\textsuperscript{-1} - please note this value is correct to the appropriate sig figs
\end{enumerate}

\hypertarget{subsec:detunitsans}{%
\subsection{Determining units of variables in equations answers}\label{subsec:detunitsans}}

\begin{enumerate}
\def\labelenumi{\arabic{enumi}.}
\item
  \begin{enumerate}
  \def\labelenumii{\alph{enumii}.}
  \item
    \begin{itemize}
    \tightlist
    \item
      kg m\textsuperscript{2} s\textsuperscript{-2} K\textsuperscript{-1} mol\textsuperscript{-1} (this is more ususally written as J K\textsuperscript{-1} mol\textsuperscript{-1})
    \end{itemize}
  \item
    \begin{itemize}
    \tightlist
    \item
      11.3 mm Hg (1.50 kPa)
    \end{itemize}
  \end{enumerate}
\end{enumerate}

\begin{center}\rule{0.5\linewidth}{0.5pt}\end{center}

\begin{enumerate}
\def\labelenumi{\arabic{enumi}.}
\setcounter{enumi}{1}
\tightlist
\item
  kg m s\textsuperscript{-1}
\end{enumerate}

\begin{center}\rule{0.5\linewidth}{0.5pt}\end{center}

\begin{enumerate}
\def\labelenumi{\arabic{enumi}.}
\setcounter{enumi}{2}
\tightlist
\item
  J K\textsuperscript{-4} mol\textsuperscript{-1}
\end{enumerate}

\begin{center}\rule{0.5\linewidth}{0.5pt}\end{center}

\begin{enumerate}
\def\labelenumi{\arabic{enumi}.}
\setcounter{enumi}{3}
\tightlist
\item
  kg m\textsuperscript{3} s\textsuperscript{-4} A\textsuperscript{-2} or N m\textsuperscript{2} C\textsuperscript{-2}
\end{enumerate}

\hypertarget{ch:Workshop2}{%
\chapter{Week 2}\label{ch:Workshop2}}

\hypertarget{sec:Prelim2}{%
\section{Preliminary infomation}\label{sec:Prelim2}}

\hypertarget{sec:rulespowers}{%
\subsection{Rules of powers and exponents}\label{sec:rulespowers}}

\begin{equation}
m^a \times m^b = m^{a+b}
\label{eq:combpowermult}
\end{equation}

\begin{equation}
\frac{p^a}{p^b} = p^{a - b}
\label{eq:combpowerdiv}
\end{equation}

\begin{equation}
\left(q^a\right)^b =  q^{a\times b}
\label{eq:combpowerraise}
\end{equation}

\emph{Anything} raised to the power 0 is equal to 1.

\begin{equation*}
x^0=1
\end{equation*}

Roots may be expressed as fractional powers:

\begin{equation}
\sqrt[n]{x}=x^{\frac{1}{n}}
\label{eq:fracpower}
\end{equation}

When we see negative powers it is the same as the inverser of the positive power.

\begin{equation}
x^{-n}=x^{\frac{1}{x^n}}
\label{eq:negpower}
\end{equation}

\hypertarget{sec:rulelog}{%
\subsection{Rules of logs}\label{sec:rulelog}}

Logs are the inverse function of exponents, and can have many bases:

When we use `natural logs' we use the terminology \(\ln\), a natural log is the inverse of `\(e\)'.

\begin{equation}
x = \ln e^x
\label{eq:natlog}
\end{equation}

Other logs are ususally marked with the \emph{base}, however if no base is indicated it should be considered that this is \(\log_{10}\).

\begin{equation}
x = \log_{10} 10^x
\label{eq:10log}
\end{equation}

When combining logs (these rules are the same regardless of base):

\begin{equation}
\log_x A + \log_x B = \log_x (AB)
\label{eq:logadd}
\end{equation}

\begin{equation}
\log_x A  - \log_x B = \log_x \left(\frac{A}{B}\right)
\label{eq:logsub}
\end{equation}

\begin{equation}
\log_x (A^n)= n \log_x A
\label{eq:logpower}
\end{equation}

If we want to change the bases of logs (such as in the Beer-Lambert law):

\begin{equation}
log_b A = \frac{\log_x A}{\log_x b}
\label{eq:convpower}
\end{equation}

\hypertarget{sec:Questions2}{%
\section{Questions}\label{sec:Questions2}}

\hypertarget{sec:logpractice}{%
\subsection{Simple log practice}\label{sec:logpractice}}

Evaluate the following expressions:

\begin{enumerate}
\def\labelenumi{\arabic{enumi}.}
\tightlist
\item
  \(\log_{10} 10^6\)
\item
  \(\log_{10} 10^{-5}\)
\item
  \(\log_{10} (5^4 \times 3^{-2})\)
\item
  \(\ln {\pi 6^2}\)
\item
  \(e^{\log_e x}=\ln y\)
\end{enumerate}

\hypertarget{sec:logrearrange}{%
\subsection{Rearranging equations}\label{sec:logrearrange}}

\begin{enumerate}
\def\labelenumi{\arabic{enumi}.}
\tightlist
\item
  Rearrange the following to make the highlighted term the subject:

  \begin{enumerate}
  \def\labelenumii{\alph{enumii}.}
  \tightlist
  \item
    \(\Delta S = k_B \ln W\), \(W\)
  \item
    \(\Delta S = nR \ln \frac{V_f}{V_i}\), \(V_f\)
  \item
    \(\nu =\frac{1}{2 \pi} \left(\frac{k}{\mu}\right)^\frac{1}{2}\)
  \item
    \(\ln K = \frac{nFE}{RT}\), \(E\)
  \item
    \(\ln K'- \ln K = \frac{\Delta H}{R}\left(\frac{1}{T}-\frac{1}{T'}\right)\), \(\Delta H\)
  \end{enumerate}
\end{enumerate}

\begin{center}\rule{0.5\linewidth}{0.5pt}\end{center}

\begin{enumerate}
\def\labelenumi{\arabic{enumi}.}
\setcounter{enumi}{1}
\tightlist
\item
  The integrated rate equation for a first order reaction is \([A]=[A]_0 e^{-kt}\).

  \begin{enumerate}
  \def\labelenumii{\alph{enumii}.}
  \tightlist
  \item
    Rearrange this equation in order to make \(k\) the subject.
  \item
    What units must k have?
  \end{enumerate}
\end{enumerate}

\hypertarget{sec:pKa}{%
\subsection{pK\_a question.}\label{sec:pKa}}

The degree of dissociation of an acid, \(\alpha\) is related to the acid dissociation constant, \(K_a\) and the concentration of the acid, c, as shown in equation \eqref{eq:aciddissociation}

\begin{equation}
K_a=\frac{\alpha ^2 c}{1-\alpha}
\label{eq:aciddissociation}
\end{equation}

Determine the pH of hydrofluoric acid solutions (p\(K_a= 3.1\)) when the concentration of acid is:
a. 1.00 M
a. 2.50 mM

\hypertarget{sec:pH}{%
\subsection{pH question.}\label{sec:pH}}

HCl fully dissociates in water. If 5 cm\textsuperscript{3} (measured using a glass pipette) of 38\% w/w HCl solution (\(\rho\) = 1.189 kg dm\textsuperscript{−3}) is `added to 20 cm\textsuperscript{3} water'made up' in a 25 cm\textsuperscript{3} standard flask.

\begin{enumerate}
\def\labelenumi{\alph{enumi}.}
\tightlist
\item
  What is the pH of the resulting solution?
\item
  What mass of NaOH is required to neutralise the resulting solution?
\end{enumerate}

\emph{Hint: w/w means weight-weight, i.e.~the number of g in 100 g. In this case 38 g of HCl in 100 g total of mixture.}

\emph{Hint: you will need to think about units and rearranging equations from Week \ref{ch:Workshop1}.}

\hypertarget{sec:Answers2}{%
\section{Answers}\label{sec:Answers2}}

  \bibliography{book.bib,packages.bib}

\end{document}
